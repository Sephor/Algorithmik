\documentclass{scrartcl}
\usepackage{amsmath, amssymb, amsthm}
\usepackage{ngerman}
\usepackage[utf8]{inputenc}
\usepackage[T1]{fontenc}
\usepackage{geometry}
\usepackage{color}
\usepackage{graphicx}
\usepackage{algorithm}
\usepackage{hyperref}
%%\usepackage{algpseudocode}
\renewcommand*\thealgorithm{}
\usepackage{algorithmic}
\geometry{top=24mm,textheight=245mm,textwidth=160mm,heightrounded,
right=27mm,head=14.5pt}


\renewcommand{\labelenumi}{\alph{enumi})}
\setlength{\parindent}{0mm}

\DeclareMathOperator{\rev}{rev}

\newcommand{\loes}[2]{\vspace{0.5cm}\\\medskip\noindent{{\bf Lösung zu Aufgabe #1}\\
{#2}}}
\newcommand{\teil}[2]{\vspace{0.3cm}\\\medskip\noindent{{\bf#1)}\\
{#2}}}
\newcommand{\bew}[2]{{\it ZZ. }{#1}\\
{\begin{proof}[Beweis] {#2} \end{proof}}}


\begin{document}
\pagestyle{plain}


\noindent
\begin{minipage}{0.66\textwidth}
Übungen zur Vorlesung\\
Einführung in die Algorithmik WS 15/16\\
~\\
%%%%%%%%%%%%%%%%%% HIER NAME EINTRAGEN! %%%%%%%%%%%%%%%%%%%%%%%%%%%%%
\textbf{Moritz Eissenhauer}
%%%%%%%%%%%%%%%%%%%%%%%%%%%%%%%%%%%%%%%%%%%%%%%%%%%%%%%%%%%%%%%%%%%%%
\end{minipage}
~
\begin{minipage}{0.30\textwidth}
Hasso-Plattner-Institut\\
Potsdam\\
\today
\end{minipage}


\begin{center}
 \huge \bf Lösung zu Bonusblatt 2
\end{center}

\loes{1}{
\teil{a}{
    \bew{Für alle A gibt es ein lokales Minimum.}{
        Da alle $n^2$ Elemente paarweise verschieden sind, gibt es ein absolutes Minimum. Dies ist kleiner als alle anderen Elemente in A also insbesondere auch kleiner als seine Nachbarn und damit ein lokales Minimum.
    }
}
\teil{b}{
\begin{algorithm}
    \floatname{algorithm}{LokalesMinimum(A, m)}
    \caption{~}
    {\bf Input: }{$n \times m$ Array $A$ mit paarweise verschiedenen Elementen (0-basiert), Anzahl der Spalten $m$}\\
    {\bf Output: }{Ein lokales Minimum in $A$}
    \begin{algorithmic}[1]
        \STATE $midColumn \leftarrow \lfloor \frac{m}{2} \rfloor$
        \STATE $min \leftarrow $ Index des Minimum in Spalte $midColumn$
        \IF{$(A[min][midColumn - 1] > A[min][midColumn]) \wedge (A[min][midColumn + 1] > A[min][midColumn])$}
        \RETURN $A[min][midColumn]$
        \ENDIF
        \IF{$A[min][midColumn - 1] < A[min][midColumn + 1]$}
            \STATE $B \leftarrow$ Teilarray von A mit Spalten $0$ bis exklusive $midColumn$
            \STATE $newM \leftarrow midColumn$
        \ELSE
            \STATE $B \leftarrow$ Teilarray von A mit Spalten $midColumn + 1$ bis $m - 1$
            \STATE $newM \leftarrow m - midColumn - 1$
        \ENDIF
        \RETURN $LokalesMinimum(B, newM)$
    \end{algorithmic}
\end{algorithm}}
Dabei sei ein Element auserhalb des Arrays (z.B. $A[0][-1]$) immer größer als Elemente im Array.
}
}


\end{document}
