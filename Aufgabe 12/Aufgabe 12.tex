\documentclass{scrartcl}
\usepackage{amsmath, amssymb, amsthm}
\usepackage{ngerman}
\usepackage[utf8]{inputenc}
\usepackage[T1]{fontenc}
\usepackage{geometry}
\usepackage{color}
\usepackage{graphicx}
\usepackage{algorithm}
\usepackage{hyperref}
%%\usepackage{algpseudocode}
\renewcommand*\thealgorithm{}
\usepackage{algorithmic}
\geometry{top=24mm,textheight=245mm,textwidth=160mm,heightrounded,
right=27mm,head=14.5pt}


\renewcommand{\labelenumi}{\alph{enumi})}
\setlength{\parindent}{0mm}

\DeclareMathOperator{\rev}{rev}

\newcommand{\loes}[2]{\vspace{0.5cm}\\\medskip\noindent{{\bf Lösung zu Aufgabe #1}\\
{#2}}}
\newcommand{\teil}[2]{\vspace{0.3cm}\\\medskip\noindent{{\bf#1)}\\
{#2}}}
\newcommand{\bew}[2]{
{\textbf{\textit{ZZ.}} }{#1}\\
{\begin{proof}[{\textbf{\textit{Beweis}}}] {#2} \end{proof}}}
\newcommand{\spliteq}[1]{
    \begin{equation}
    \begin{split}
    {#1}
    \end{split}
    \end{equation}
}


\begin{document}
\pagestyle{plain}


\noindent
\begin{minipage}{0.66\textwidth}
Übungen zur Vorlesung\\
Einführung in die Algorithmik WS 15/16\\
~\\
%%%%%%%%%%%%%%%%%% HIER NAME EINTRAGEN! %%%%%%%%%%%%%%%%%%%%%%%%%%%%%
\textbf{Moritz Eissenhauer}
%%%%%%%%%%%%%%%%%%%%%%%%%%%%%%%%%%%%%%%%%%%%%%%%%%%%%%%%%%%%%%%%%%%%%
\end{minipage}
~
\begin{minipage}{0.30\textwidth}
Hasso-Plattner-Institut\\
Potsdam\\
\today
\end{minipage}


\begin{center}
%%%%%%%%%%%%%%%%%% HIER NUMMER EINTRAGEN! %%%%%%%%%%%%%%%%%%%%%%%%%%%%
 \huge \bf Lösung zu Übungsblatt 12
 %%%%%%%%%%%%%%%%%%%%%%%%%%%%%%%%%%%%%%%%%%%%%%%%%%%%%%%%%%%%%%%%%%%%%
\end{center}

\loes{1}{
\teil{a}{
  \bew{Es gibt höchstens einen \textit{absoluten Gewinner} (aG).}{
    Annahme: Seien $a, b \in V$ zwei aG in $G_{EM}$.\\
    Da $a$ und $b$ gegeneinander angetreten haben muss eine Kante $e \in E$ existieren, welche $a$ und $b$ verbindet. $e$ muss aber entweder $(a, b)$ oder $(b, a)$ sein. Beide Fälle stehen im widerspruch zur Annahme.\\
    $\Rightarrow$ Es kann keine zwei (und damit auch nicht mehr) aG geben.
  }
}
\teil{b}{
  Eine Mannschaft ist genau dann der aG wenn ihre Adjazenzliste leer ist.\\
  Man brauch also lediglich über alle Manschaften itterieren und prüfen ob ihre Adjazenzlisten leer sind. Iste eine liste leer so gibt man die zugehöhrige Mannschaft aus.\\
  Die Laufzeit um eine liste auf Leerheit zu prüfen ist konstant, da man dies für jede Mannschaft machen muss ist die Gesamtlaufzeit $\in \mathcal{O}(n)$.
}
}

\loes{2}{
\begin{algorithm}
  \floatname{algorithm}{longest}
  \caption{~}
  \textbf{Input: }Adjazenzliste des Baumes\\
  \textbf{Output: }Endpunkte eines längsten Pfades
  \begin{algorithmic}[1]
    \STATE $s \leftarrow$ beliebiger Knoten aus Baum
    \STATE $a \leftarrow$ letzter Knoten bei Breitensuche mit Start $s$
    \STATE $b \leftarrow$ letzter Knoten bei Breitensuche mit Start $a$
    \RETURN $a, b$
  \end{algorithmic}
\end{algorithm}\\
\textbf{Korrektheit:}\\
  $a$ ist der von $s$ am weitesten entfernte Knoten. Es sind zwei Fälle zu unterscheiden:\\
  \textbf{Fall 1:} $s$ liegt auf einem längsten Pfad\\
  $a$ muss dadurch Startpunkt (oder Endpunkt) eines längsten Pfades sein. Falls es einen längsten Pfad gibt, der bei $x \neq a$ startet, bei $y \neq a$ endet und über $s$ geht, dann gibt es auch den Pfad der bei $a$ startet und über $s$ bei $y$ oder $x$ endet. Dieser ist aber auch ein längster Pfad, da der Pfad von $a$ nach $s$ mindestens so lang ist wie der von $x$ nach $s$ bzw. der von $s$ nach $y$.\\
  Da es einen längsten Pfad gibt der bei $a$ startet und der Pfad $a$ nach $b$ der längste Pfad ist der bei $a$ startet mus dies auch ein längster Pfad im Baum sein.\\
  \textbf{Fall 2:} $s$ liegt nicht auf einem längsten Pfad\\
  ... hier fehlt mir der Ansatz.
  \\
  \textbf{Laufzeit:}\\
  Sei $n$ die Anzahl der Knoten und $m$ die der Kanten. Für einen Baum gilt $m = n - 1$. Die Laufzeit der Breitensuche ist damit $\in \Theta(n)$. Die Gesamtlaufzeit ist folglich $\in \mathcal{O}(n)$.
}


\end{document}
