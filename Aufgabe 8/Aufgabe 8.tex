\documentclass{scrartcl}
\usepackage{amsmath, amssymb, amsthm}
\usepackage{ngerman}
\usepackage[utf8]{inputenc}
\usepackage[T1]{fontenc}
\usepackage{geometry}
\usepackage{color}
\usepackage{graphicx}
\usepackage{algorithm}
\usepackage{hyperref}
%%\usepackage{algpseudocode}
\renewcommand*\thealgorithm{}
\usepackage{algorithmic}
\geometry{top=24mm,textheight=245mm,textwidth=160mm,heightrounded,
right=27mm,head=14.5pt}


\renewcommand{\labelenumi}{\alph{enumi})}
\setlength{\parindent}{0mm}

\DeclareMathOperator{\rev}{rev}

\newcommand{\loes}[2]{\vspace{0.5cm}\\\medskip\noindent{{\bf Lösung zu Aufgabe #1}\\
{#2}}}
\newcommand{\teil}[2]{\vspace{0.3cm}\\\medskip\noindent{{\bf#1)}\\
{#2}}}
\newcommand{\bew}[2]{
{\textbf{\textit{ZZ.}} }{#1}\\
{\begin{proof}[{\textbf{\textit{Beweis}}}] {#2} \end{proof}}}
\newcommand{\spliteq}[1]{
    \begin{equation}
    \begin{split}
    {#1}
    \end{split}
    \end{equation}
}


\begin{document}
\pagestyle{plain}


\noindent
\begin{minipage}{0.66\textwidth}
Übungen zur Vorlesung\\
Einführung in die Algorithmik WS 15/16\\
~\\
%%%%%%%%%%%%%%%%%% HIER NAME EINTRAGEN! %%%%%%%%%%%%%%%%%%%%%%%%%%%%%
\textbf{Moritz Eissenhauer}
%%%%%%%%%%%%%%%%%%%%%%%%%%%%%%%%%%%%%%%%%%%%%%%%%%%%%%%%%%%%%%%%%%%%%
\end{minipage}
~
\begin{minipage}{0.30\textwidth}
Hasso-Plattner-Institut\\
Potsdam\\
\today
\end{minipage}


\begin{center}
%%%%%%%%%%%%%%%%%% HIER NUMMER EINTRAGEN! %%%%%%%%%%%%%%%%%%%%%%%%%%%%
 \huge \bf Lösung zu Übungsblatt 8
 %%%%%%%%%%%%%%%%%%%%%%%%%%%%%%%%%%%%%%%%%%%%%%%%%%%%%%%%%%%%%%%%%%%%%
\end{center}

\loes{1}{
  \textbf{Algorithmus:}\\
  \begin{algorithm}
    \floatname{algorithm}{Tiefe}
    \caption{~}
    \textbf{Input: }$n$ Zeitintervalle mit Start- und Endzeitpunkten\\
    \textbf{Output: }Tiefe $d$
    \begin{algorithmic}[1]
      \STATE $Z \leftarrow$ alle Zeitpunkte als Zahl, mit markierung ob sie Start- oder Endzeiten sind.
      \STATE Sortiere $Z$ aufsteigend nach Zeitpunkt
      \STATE $max \leftarrow 0$
      \STATE $current \leftarrow 0$
      \FOR{$zeitpunkt$ \textbf{in} $Z$}
        \IF{$zeitpunkt$ ist Startzeit}
          \STATE $current++$
        \ELSE
          \STATE $current--$
        \ENDIF
        \IF{$current>max$}
          \STATE $max \leftarrow current$
        \ENDIF
      \ENDFOR
      \RETURN max
    \end{algorithmic}
  \end{algorithm}\\
  \textbf{Laufzeit:}\\
  Alle Operationen in der Schleife haben konstante Laufzeit, und die Schleife läuft $2n$ mal (es gibt insgesamt $2n$ Zeitpunkte). Insgesamt hat die Schleife eine Laufzeit von $\mathcal{O}(n)$. Das Initialisieren von $Z$ braucht hat ebenfalls eine Laufzeit von $\mathcal{O}(n)$. Zeile 3 und 4 haben jeweils eine $\mathcal{O}(1)$ Laufzeit. Das Sortieren in Zeile 2 braucht $\mathcal{O}(n \log n)$. Die gesamtlaufzeit ist damit $\mathcal{O}(n \log n)$.
}

\loes{2}{
\teil{a}{
  \spliteq{
    d(1) = 3, &\ t(1) = 1\\
    d(2) = 2, &\ t(2) = 2
  }
  $d(i)$ sei die Deadline, $t(i)$ die benötigte Zeit des $i$-ten Jobs.\\
  Lösung des Algorithmus: 1, 2\\
  $\hookrightarrow$ Maximale Überschreitung der Deadline von 1 (Job 2 wird bei $t=3$ fertig).\\
  Optimale Lösung: 2, 1\\
  $\hookrightarrow$ Keine Deadine wird verletzt.\\
}
\teil{b}{
  \spliteq{
    d(1) = 4, &\ t(1) = 3 \Rightarrow s(1) = 1\\
    d(2) = 3, &\ t(2) = 1 \Rightarrow s(2) = 2
  }
  $s(i)$ist die Slack-time vom $i$-ten Job.\\
  Lösung des Algorithmus: 1, 2\\
  $\hookrightarrow$ Maximale Überschreitung der Deadline von 1. (Job 2 wird bei $t=4$ fertig)\\
  Optimale Lösung: 2, 1\\
  $\hookrightarrow$ Keine Deadine wird verletzt.\\
}
}
\loes{3}{
\textbf{Beobachtung:}\\
Wenn die Teilnehmer in der Reihnfolge $\sigma=1, 2, 3, ..., n$ starten dann lässt sich die Gesamtzeit berechnen mit:\\
\spliteq{
  T(\sigma) = \max \left\{ \sum_{i=1}^k p_i + r_k | 1 \leq k \leq n \right\}
}
Oder: Der $k$-te Teilnehmer muss die Paddelzeit all seine Vorgänger abwarten und dann noch selbst paddeln und ins Ziel radeln.
\teil{a}{
  Zu Strategie \textbf{(1)}:\\
  $n = 2$
  \spliteq{
    p_1 = 3, &\ r_1 = 1 \Rightarrow t_1 = 4\\
    p_2 = 1, &\ r_2 = 2 \Rightarrow t_2 = 3
  }
  Lösung des Algorithmus: $\sigma = 1, 2$\\
  $\hookrightarrow T(\sigma) = \max \left\{ 6, 4 \right\} = 6$\\
  Optimale Lösung: $\sigma = 2, 1$\\
  $\hookrightarrow T(\sigma) = \max \left\{ 5, 3 \right\} = 5$\\
  \\
  Zu Strategie \textbf{(2)}:\\
  $n = 2$
  \spliteq{
    p_1 = 1, &\ r_1 = 1\\
    p_2 = 2, &\ r_2 = 2
  }
  Lösung des Algorithmus: $\sigma = 1, 2$\\
  $\hookrightarrow T(\sigma) = \max \left\{ 5, 2 \right\} = 5$\\
  Optimale Lösung: $\sigma = 2, 1$\\
  $\hookrightarrow T(\sigma) = \max \left\{ 4, 4 \right\} = 4$
}
}
\loes{4}{
\teil{a}{
  \spliteq{
    k = 2 & \\
    n = 2 & \\
    w(1) = 1, &\ g(1) = 0.1\\
    w(2) = 10, &\ g(2) = 2
  }
  Algorithmus \textbf{(1)} wählt nur Geschenk 1 $\Rightarrow ALG_1 = 1$, Algorithmus \textbf{(2)} wählt nur Geschenk 2 $\Rightarrow ALG_2 = 10$.
}
\teil{b}{
  \bew{$\frac{OPT}{ALG_1}$ kann beliebig groß werden.}{
    Sei $k>0$ beliebig, fest, sei $n=2$.\\
    Wähle die Geschenke so, dass für ein beliebiges $x>0$ folgendes gilt:\\
    $w(1) = 1,\ g(2) = k,\ w(2) = x,\ g(1) = \frac{k}{2x}$.\\
    Sei $e(i)$ die effizienz (Wert pro Gewicht) des $i$-ten Geschenkes.\\
    Aus den gegebenen Angaben lässt sich die Effizienz der ersten beiden Geschenke bestimmen als $e(1) = 2 \frac{x}{k}$, $e(2) = \frac{x}{k}$. Offensichtlich ist die Effizienz des ersten Geschenkes höher als die des Zweiten. Der Algorithmus wählt also zuerst Geschenk 1. Da $g(1) = \frac{k}{2x} > 0$ und $g(2) = k$ Lädt der Algorithmus nur das Geschenk 1 auf den Schlitten. $ALG_1$ ist damit $1$. Es gibt aber immer nur eine alternative Lösung $SubOPT = x$ die nur das zweite Geschenk wählt.\\
    \spliteq{
      ALG_1 = 1 \Rightarrow & \frac{OPT}{ALG_1} = OPT \geq SubOPT = x\\
      \Rightarrow & \frac{OPT}{ALG_1} \geq x
    }
    Da $x$ beliebig groß gewählt werden kann, kann auch $\frac{OPT}{ALG_1}$ beliebig groß werden.
  }
}
\teil{c}{
  \bew{$\frac{OPT}{ALG_2}$ kann beliebig groß werden.}{
    Sei $k>0$ beliebig, fest, sei $n=3$.\\
    Wähle die Geschenke so, dass für ein beliebiges $x>0$ folgendes gilt:\\
    $w(1) = 1,\ g(1) = \frac{k}{3x},\ w(2) = 4x,\ g(2) = 2k,\ g(3) = k,\ w(3) = x$.\\
    Für die Effizienz der Geschenke gilt:\\
    $e(1) = 3 \frac{x}{k},\ e(2) = 2 \frac{x}{k},\ e(3) = \frac{x}{k}$\\
    Die Geschenkmenge $S={1}$, das erste Geschenk was nicht gewählt werden kann ist $x=2$. Da $x$ den Schlitten überlädt wählt der Algorithmus $S \Rightarrow ALG_2 = w(1) = 1$. Eine alternative Lösung $SubOPT$ wäre nur das Geschenk 3 $\Rightarrow SubOPT = w(3) = x$.\\
    \spliteq{
      ALG_2 = 1 \Rightarrow & \frac{OPT}{ALG_2} = OPT \geq SubOPT = x\\
      \Rightarrow & \frac{OPT}{ALG_2} \geq x
    }
    Da $x$ beliebig groß gewählt werden kann, kann auch $\frac{OPT}{ALG_2}$ beliebig groß werden.
  }
  Das lößt zwar die Aufgabe nicht, zeigt aber das die Aufgabe auch nicht lösbar ist.
}
}

\end{document}
