\documentclass{scrartcl}
\usepackage{amsmath, amssymb, amsthm}
\usepackage{ngerman}
\usepackage[utf8]{inputenc}
\usepackage[T1]{fontenc}
\usepackage{geometry}
\usepackage{color}
\usepackage{graphicx}
\usepackage{algorithm}
\usepackage{hyperref}
%%\usepackage{algpseudocode}
\renewcommand*\thealgorithm{}
\usepackage{algorithmic}
\geometry{top=24mm,textheight=245mm,textwidth=160mm,heightrounded,
right=27mm,head=14.5pt}


\renewcommand{\labelenumi}{\alph{enumi})}
\setlength{\parindent}{0mm}

\DeclareMathOperator{\rev}{rev}

\newcommand{\loes}[2]{\vspace{0.5cm}\\\medskip\noindent{{\bf Lösung zu Aufgabe #1}\\
{#2}}}
\newcommand{\teil}[2]{\vspace{0.3cm}\\\medskip\noindent{{\bf#1)}\\
{#2}}}
\newcommand{\bew}[2]{
{\textbf{\textit{ZZ.}} }{#1}\\
{\begin{proof}[{\textbf{\textit{Beweis}}}] {#2} \end{proof}}}
\newcommand{\spliteq}[1]{
    \begin{equation}
    \begin{split}
    {#1}
    \end{split}
    \end{equation}
}


\begin{document}
\pagestyle{plain}


\noindent
\begin{minipage}{0.66\textwidth}
Übungen zur Vorlesung\\
Einführung in die Algorithmik WS 15/16\\
~\\
%%%%%%%%%%%%%%%%%% HIER NAME EINTRAGEN! %%%%%%%%%%%%%%%%%%%%%%%%%%%%%
\textbf{Moritz Eissenhauer}
%%%%%%%%%%%%%%%%%%%%%%%%%%%%%%%%%%%%%%%%%%%%%%%%%%%%%%%%%%%%%%%%%%%%%
\end{minipage}
~
\begin{minipage}{0.30\textwidth}
Hasso-Plattner-Institut\\
Potsdam\\
\today
\end{minipage}


\begin{center}
%%%%%%%%%%%%%%%%%% HIER NUMMER EINTRAGEN! %%%%%%%%%%%%%%%%%%%%%%%%%%%%
 \huge \bf Lösung zu Übungsblatt 7
 %%%%%%%%%%%%%%%%%%%%%%%%%%%%%%%%%%%%%%%%%%%%%%%%%%%%%%%%%%%%%%%%%%%%%
\end{center}

\loes{3}{
\teil{a}{
    Die Amorisierten Kosten seien wie folgt:\\
    \spliteq{
        1 & $ für $  0 \rightarrow 1\\
        2 & $ für $  1 \rightarrow 2\\
        0 & $ für $  2 \rightarrow 0
    }
    Wobei $0 \rightarrow 1$ die Änderung eines Trit von $0$ auf $1$ sei.\\
    Die Tatsächlichen Kosten der $i$-ten Inkrementierung ergeben sich aus $t_i = e_i+z_i+n_i$. $e_i$ ist die Anzahl der Tritänderung von $0$ auf $1$, $z_i$ von $1$ auf $2$ und $n_i$ von $2$ auf $0$. Mit den oben angegebenen Kosten erhält man für die amorisierten Kosten der $i$-ten Inkrementierung $a_i = 1e_i+2z_z+0n_i$.\\
    Das Guthaben entspricht immer der Anzahl der $2$en, weil genau dann ein Guthaben hinzugefügt wird wenn eine neue $2$ entsteht, und genau dann ein Guthaben abgezogen wird wenn eine $2$ zur $0$ wird. Es steht also immer mindestens genug Guthaben zu verfügung um alle $2$en zu $0$en zu ändern. Die Gesamtkosten um $n$ mal zu inkrementieren sind damit abschätzbar durch:\\
    \spliteq{
        T(n) = \sum_{i=1}^n t_i \leq \sum_{i=1}^n a_i
    }
    Bei jeder Inkrementierung wird aber höchstens ein Trit von von $1$ auf $2$ geändert oder ein Trit von $0$ auf $1$. Das heißt $\forall i \leq n: a_i \leq 2$. Für die Gesamtkosten ergibt sich somit $T(n) \leq \sum_{i=1}^n 2 = 2n \in \mathcal{O}(n)$.
}
\teil{b}{
    Die Addition von $11_2$ is Äquivalent zu erst $10_2$ und dann $1_2$ addieren. Die Amorisierten Kosten um $10_2$ zu addieren sind höchstens gleich hoch wie die um $1_2$ zu addieren (man ignoriert einfach die letzte stell und wendet genau das gleiche Verfahren an). Die Kosten das $i$-te mal $3$ zu addieren sind also $a_{i,3} = a_{i,2} + a_{i,1} = 2 a_{i,1} = 4$ wobei $a_{i,1}$ die kosten sind um das $i$-te mal $1$ zu addieren.
}
}
\loes{4}{
\teil{a}{
    Wähle man wähle $\Phi(i)$ als:\\
    \spliteq{
        \Phi(i)= c_i - \lfloor \frac{|A_i|}{2} \rfloor
    }
    $c_i$ ist die länge des dynamischen Arrays am \textbf{anfang} der $i$-ten Einfügeoperation. $|A_i|$ ist die länge von $A$ am \textbf{ende} der $i$-ten Einfügeoperation.
    Da initial $c_i = 0$ und $|A_i| = 1$ gilt $\Phi(0) = 0$. Sei $\Delta \Phi(i) = \Phi(i) - \Phi(i-1)$, dann gilt für die amorisierten Kosten $a_i = t_i + \Delta \Phi(i)$. Wenn man nur nacheinander einfügt bleibt $c_i \geq \lfloor \frac{|A_i|}{2} \rfloor$ und $\Phi(i)$ ist somit immer positivy. Daraus folgt, dass wir die tatsächlichen Kosten abschätzen können mit:\\
    \spliteq{
        T(n) = \sum_{i=1}^n t_i \leq \sum_{i=1}^n a_i
    }
    Angenommen die $i$-te Operation muss $k_i$ Elemente kopieren. Dann sind die tatsächlihen Kosten $t_i = k_i + 1$ weil auch noch das neue Element geschrieben werden muss. Wenn aber ein neues, doppelt so großes Array angelegt wird dann wird auch das Potential um $k_i$ kleiner, da dann $c_i = \frac{|A_i|}{2} \rfloor$ (die Elemente passen jetzt alle in die linke Hälfte des Array). Für Das Potential gilt:\\
    \spliteq{
        &\Phi(i) \leq \Phi(i-1) + 1 - k_i \\
        \Leftrightarrow & \Delta \Phi(i) \leq 1 - k_i
    }
    Die amorisiertn Kosten sind damit\\
    \spliteq{
        a_i & = t_i + \Delta \Phi(i)\\
            & \leq k_i + 1 + 1 - k_i\\
            & = 2
    }
    und für die Gesamtkosten gilt damit
    \spliteq{
        T(n) \leq \sum_{i=1}^n a_i \leq 2n \in \mathcal{O}(n)
    }
}
\teil{b}{
    Wenn $2^k$ viele Elemente eingefügt wurden, dann braucht das Einfügen  $e = 2^k c_c + c_e$ Kosten, wobei $c_c$ die Kosten zum kopieren und $c_e$ die zum einfügen sind. Wnn man dannach gleich wieder löscht, dann sind die Kosten $l = 2^k c_c + c_d$ wobei $c_d$ die kosten zum Löschen sind. Wenn $t_i$ die kosten für das $i$-te Einfügen sind, dann sind die Kosten genau dann maximal wenn\\
    \spliteq{
        T_n(k)=\sum_{i=1}^{2^k} t_i + \frac{(n-2^k)}{2} (e + l)
    }
    maximal ist. Wenn man $k$ findet dann ist die Sequenz $2^k$ mal append gefolgt von $ \frac{(n-2^k)}{2}$ mal append und removeLast abgewechselt. \footnote{Irgendwas muss ich übersehen haben, das es einfacher macht}
}
}


\end{document}
