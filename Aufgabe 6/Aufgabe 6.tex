\documentclass{scrartcl}
\usepackage{amsmath, amssymb, amsthm}
\usepackage{ngerman}
\usepackage[utf8]{inputenc}
\usepackage[T1]{fontenc}
\usepackage{geometry}
\usepackage{color}
\usepackage{graphicx}
\usepackage{algorithm}
\usepackage{hyperref}
%%\usepackage{algpseudocode}
\renewcommand*\thealgorithm{}
\usepackage{algorithmic}
\geometry{top=24mm,textheight=245mm,textwidth=160mm,heightrounded,
right=27mm,head=14.5pt}


\renewcommand{\labelenumi}{\alph{enumi})}
\setlength{\parindent}{0mm}

\DeclareMathOperator{\rev}{rev}

\newcommand{\loes}[2]{\vspace{0.5cm}\\\medskip\noindent{{\bf Lösung zu Aufgabe #1}\\
{#2}}}
\newcommand{\teil}[2]{\vspace{0.3cm}\\\medskip\noindent{{\bf#1)}\\
{#2}}}
\newcommand{\bew}[2]{{\textbf{\textit{ZZ.}} }{#1}\\
{\begin{proof}[{\textbf{\textit{Beweis}}}] {#2} \end{proof}}}
\newcommand{\spliteq}[1]{
\begin{equation}
\begin{split}
{#1}
\end{split}
\end{equation}
}

\begin{document}
\pagestyle{plain}


\noindent
\begin{minipage}{0.66\textwidth}
Übungen zur Vorlesung\\
Einführung in die Algorithmik WS 15/16\\
~\\
%%%%%%%%%%%%%%%%%% HIER NAME EINTRAGEN! %%%%%%%%%%%%%%%%%%%%%%%%%%%%%
\textbf{Moritz Eissenhauer}
%%%%%%%%%%%%%%%%%%%%%%%%%%%%%%%%%%%%%%%%%%%%%%%%%%%%%%%%%%%%%%%%%%%%%
\end{minipage}
~
\begin{minipage}{0.30\textwidth}
Hasso-Plattner-Institut\\
Potsdam\\
\today
\end{minipage}


\begin{center}
%%%%%%%%%%%%%%%%%% HIER NUMMER EINTRAGEN! %%%%%%%%%%%%%%%%%%%%%%%%%%%%
 \huge \bf Lösung zu Übungsblatt 6
%%%%%%%%%%%%%%%%%%%%%%%%%%%%%%%%%%%%%%%%%%%%%%%%%%%%%%%%%%%%%%%%%%%%%%
\end{center}

\loes{2}{
\teil{a}{
    \bew{Sort ist Korrekt (für $j > i \geq 0$)}{
        Seien $x$ und $y$ beliebige Elemente in $A$ mit $i \leq ind(x) \leq j$ und $i \leq ind(y) \leq j$, wobei $ind(x)$ der Index von $x$ in $A$ ist.\\
        Zeige: Falls $x < y$ dann ist nach Ausführung von sort $ind(x) < ind(y)$ ($\Leftrightarrow A$ ist sortiert von $i$ bis $j$). (I)\\
        {\bf Induktiv über $j-i$:}\\
        {\bf Induktions Vorraussetzung:} Falls $j-i = 1$ dann ist $j = i + 1$ und der erste \textbf{if}-Block wird betreten. Wegen den Bedingungen oben ist entweder $A[i] = x \wedge A[j] = y$ oder $A[j] = x \wedge A[i] = y$. Für den ersten Fall ist $A[j] < A[i] = False$, nichts wird getauscht und $ind(x) < ind(y)$ gilt. Für den zweiten Fall ist $A[j] < A[i] = True$, $A[i]$ und $A[j]$ werden getauscht und $ind(x) < ind(y)$ gilt.\\
        Dannach wird unmittelbar returnt, (I) gilt also für $j-i=1$.\\
        {\bf Induktions Behauptung:} (I) gilt für $j-i \leq n $, $n \geq 1$ beliebig.\\
        {\bf Induktions Schritt:} Zeige (I) gilt für $j-i \leq n \Rightarrow$ (I) gilt für $j-i \leq n+1$.\\
        Sei $j-i=n+1$.\\
        Da $n \geq 1$ gilt $j = i+1$ nicht und der \textbf{if}-Block wird nicht betreten. $l$ wird auf $j-i+1=n+1$ gesetzt. Das heißt $\lfloor \frac{l}{3} \rfloor > 1$ also ist $\lceil j-\frac{l}{3}\rceil - i \leq n$. Der Aufruf in Zeile 8 sortiert die Elemente $i$ bis inklusive $\lceil j-\frac{l}{3}\rceil$.\\
        Es gilt auch $j-\lfloor i + \frac{l}{3} \rfloor \leq n$, also sortiert Zeile 9 $\lfloor i + \frac{l}{3} \rfloor$ bis $i$. Das Teilarray von $i$ bis exklusive $\lfloor i + \frac{l}{3} \rfloor$ ist also duch 8 sortiert und $\lfloor i + \frac{l}{3} \rfloor$ bis $j$ durch 9. Insbesondere gilt auch das alle Elemente im Teilarray $j-\lfloor i + \frac{l}{3} \rfloor$ bis $j$ größer sind als alle anderen.\\
        Mit Zeile 10 wird wieder $i$ bis $j-\lfloor i + \frac{l}{3} \rfloor$ sortiert, das heißt das gesamte Array ist sortiert.
    }
}
\teil{b}{
    Für $j-i=n$ lässt sich die Laufzeit als
    \begin{equation}
        \begin{split}
            T(1) &= c\\
            T(n) &= 3T(\frac{2n}{3}) + c
        \end{split}
    \end{equation}
    Mit dem Mastertheorem Fall a erhällt man:
    \spliteq{
        & \log_ba > 2\\
        & f(n) = c \in \mathcal{O}(n^2)\\
        \Rightarrow & T(n) \in \Theta(n^{\log_ba})\\
        \Leftrightarrow & T(n) \in \Theta(n^{\log_{1,5}{3}})
    }
}
}

\loes{3}{
\teil{a}{
    \bew{Genericsort ist korrekt}{
        Für $x$, $y$ beiebige ganze Dezimalzahlen mit $k$ (beliebig, fest) stellen gilt
        \begin{equation}
            x \neq y \Rightarrow \exists 1 \leq j \leq k: x_j \neq y_j
        \end{equation}
        Für das kleinste dieser $j$ gilt dann $x_j < y_j \Rightarrow: x < y$. Da die Zahlen in absteigender Reihenfolge sortiert werden und das Sortierverfahren $X$ stabil ist, steht $x \in A$ genau dann vor $x \neq y \in A$ wenn für die kleinste Stelle $j$ an der sich $x$ und $y$ unterscheiden $x_j < y_j$ gilt.
    }
}
\teil{b}{
    Nach der ersten for-Schleife stehen in $B$ am index $i$ die Anzahl der Vorkommen von $i$ in $A$. Nach der zweiten schleife steht am Index $i$ Anzahl der Vorkommen aller $x \leq i$ in $A$ minus eins. Das is Äquivalent zu dem jeweils letzten Index im sortierten array, der den wert $i$ hat.\\
    Der Algorithmus sortiert auch korrekt (aber nicht stabil) wenn die Schleife aufsteigend läuft weil nur die reihenfolge geändert wird wann welcher Wert in $C$ geschrieben wird, aber trotzdem Die gleichen Werte an die entsprechenden Stellen geschrieben werden.
}
\teil{c}{
    Anmerkung: Es is schwierig die Reienfolge der elemente von der der Schlüssel zu unterscheiden, wenn die Elemente selbst die Schlüssel sind.\\\\
    \bew{Linearsort ist stabil}{
        In der dritten Schleife wird ein wert $x$ in $C$ zuerst and die letzte stelle aller $x$ im sortierten array geschrieben.\\
        Wenn das erste mal der Wert $x$ in $A$ gelesen wird wird er also an die Letzte Stelle aller $x$ in $C$ geschrieben. Der Index an den das nächste $x$ geschrieben wird wird dann um eins verringert. Da die Schleife rückwärts läuft wird also immer das n-letzte $x$ an die n-letzte Stelle der aller $x$ geschrieben. Die reihenfolge bleibt also gleich.
    }
}
\teil{d}{
    \bew{Es gibt im worst case $\Omega(n^2)$ Vertauschungen.}{
        Wenn für alle $i < \frac{n}{2}$ gilt $A[2i] > A[2i+1]$ dann wird zuerst beim ersten Durchlauf der äuseren Schleife bei jedem zweiten Element getauscht, also werden $\frac{n}{2}$ Vertauschungen vorgenommen. Beim jedem weiteren Durchgang wird jeweils eine Vertauschung weniger vorgenommen. Die Beschriiebene Form lässt sich für alle $l  > 1$ und $n$ herstellen.\\
        Die gesamtzahl der Vertauschungen ist also $n \frac{n}{2} \in \Omega(n^2)$.
    }
}
\teil{e}{
    Die Laufzeit von Genericsort mit Bubblesort als Subroutine $X$ ist $T_G(k, n) = kT_{Bubble}(n)$ wobei $T_{Bubble}$ die Laufzeit von Bubblesort ist. $T_G(k, n)$ ist somit ein $\Theta(kn^2)$.
}
\teil{f}{
    Die Lufzeit von Linearsort ist $f_1(n) + f_2(z) + f_3(n) + c$ wobei $f_i$ die Laufzeit für die $i$-te Schleife ist. $f_1$ und $f_3$ sind in $\Theta(n)$, $f_2$ ist in $\Theta(z)$. Damit ist die gesamtlaufzeit $\Theta(n+z)$.\\
    \bew{Für $z=cn$ gilt die Laufzeit ist in $\matcal{O}(n)$}{
        Die Laufzeit für $z$ beliebig ist in $\Theta(n+z)$, die Laufzeit für $z=cn$ ist damit in $\Theta(n+cn) = \Theta(n) \cup \mathcal{O}(n)$
    }
}
\teil{g}{
    Die Laufzeit für Genericsort mit Linearsort als Subroutine ist $T_G(k, n) = kT_{Linear}(n, z)$ mit $z=10$. $T_G(k, n)$ ist also in $\Theta(kn)$.
}
}


\end{document}
