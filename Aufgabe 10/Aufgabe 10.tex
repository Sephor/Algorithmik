\documentclass{scrartcl}
\usepackage{amsmath, amssymb, amsthm}
\usepackage{ngerman}
\usepackage[utf8]{inputenc}
\usepackage[T1]{fontenc}
\usepackage{geometry}
\usepackage{color}
\usepackage{graphicx}
\usepackage{algorithm}
\usepackage{hyperref}
%%\usepackage{algpseudocode}
\renewcommand*\thealgorithm{}
\usepackage{algorithmic}
\geometry{top=24mm,textheight=245mm,textwidth=160mm,heightrounded,
right=27mm,head=14.5pt}


\renewcommand{\labelenumi}{\alph{enumi})}
\setlength{\parindent}{0mm}

\DeclareMathOperator{\rev}{rev}

\newcommand{\loes}[2]{\vspace{0.5cm}\\\medskip\noindent{{\bf Lösung zu Aufgabe #1}\\
{#2}}}
\newcommand{\teil}[2]{\vspace{0.3cm}\\\medskip\noindent{{\bf#1)}\\
{#2}}}
\newcommand{\bew}[2]{
{\textbf{\textit{ZZ.}} }{#1}\\
{\begin{proof}[{\textbf{\textit{Beweis}}}] {#2} \end{proof}}}
\newcommand{\spliteq}[1]{
    \begin{equation}
    \begin{split}
    {#1}
    \end{split}
    \end{equation}
}


\begin{document}
\pagestyle{plain}


\noindent
\begin{minipage}{0.66\textwidth}
Übungen zur Vorlesung\\
Einführung in die Algorithmik WS 15/16\\
~\\
%%%%%%%%%%%%%%%%%% HIER NAME EINTRAGEN! %%%%%%%%%%%%%%%%%%%%%%%%%%%%%
\textbf{Moritz Eissenhauer}
%%%%%%%%%%%%%%%%%%%%%%%%%%%%%%%%%%%%%%%%%%%%%%%%%%%%%%%%%%%%%%%%%%%%%
\end{minipage}
~
\begin{minipage}{0.30\textwidth}
Hasso-Plattner-Institut\\
Potsdam\\
\today
\end{minipage}

\begin{center}
%%%%%%%%%%%%%%%%%% HIER NUMMER EINTRAGEN! %%%%%%%%%%%%%%%%%%%%%%%%%%%%
 \huge \bf Lösung zu Übungsblatt 10
 %%%%%%%%%%%%%%%%%%%%%%%%%%%%%%%%%%%%%%%%%%%%%%%%%%%%%%%%%%%%%%%%%%%%%
\end{center}

\loes{1}{
  Das Rentier sollte zuerst 1 Meter nach links laufen. Dann sollte es 2 Meter wieder nach rechts laufen, dann 4 Meter nach links, dann 8 nach rechts, etc. Also nachdem es k Meter in eine Richtung gelaufen ist sollte es 2k Meter in die Aandere laufen. Dadurch entdeckt es immer zur Hälfte der zurückgelegten Strecke neue Zaunstücke, muss also ein $\mathcal{O}(n)$ Meter zurücklegen bevor es das Loch findet.
}

\loes{2}{
\teil{a}{
  \begin{algorithm}
    \floatname{algorithm}{Veranstaltungsplan}
    \caption{~}
    \textbf{Input: }Liste der Leistungspunkte $l$, Liste der Schwierigkeitsgrade $w$, minimale LP $c$\\
    \textbf{Output: }Liste zu belegener Veranstaltungen
    \begin{algorithmic}[1]
      \STATE $Plaene \leftarrow$ Liste der Zahlen von $1$ bis $n$
      \STATE $Plaene \leftarrow$ Potenzmenge von $Plaene$
      \STATE $minWork \leftarrow +inf$
      \STATE $best \leftarrow$ leere Liste
      \FORALL{$plan$ in $Plaene$}
        \STATE $work \leftarrow 0$
        \STATE $lp \leftarrow 0$
        \FORALL{$i$ in $plan$}
          \STATE $lp \leftarrow lp + l_i$
          \STATE $work \leftarrow work + w_i$
          \IF{$work < minWork$}
            \STATE $minWork \leftarrow work$
            \STATE $best \leftarrow plan$
          \ENDIF
        \ENDFOR
      \ENDFOR
      \RETURN $best$
    \end{algorithmic}
  \end{algorithm}\\
  \textbf{Laufzeit:}\\
  Die Potenzmenge hat die Größe $2^n$, mit jeweils maximal $n$ Elementen braucht also $\Theta(n2^n)$ um erstellt zu werden. Die Äußere Schleife Itteriert über $2^n$ Elemente, und die innere über $n$. Die Schleifen brauchen also auch $\Theta(n2^n)$. Der gesamte Algorithmus hat also eine Laufzeit von $\Theta(n2^n)$.
}
\teil{b}{
  Induktiver Ansatz:\\
  \spliteq{
    best(i, c) = min(best(i-1,c),best(i-1,c-l_i)+w_i)
  }
  wobei $best(i,c)$ der beste Studienplan für die ersten $i$ Veranstaltungen mit zu erreichenden Leistungspunkten $c$ ist.\\
  \begin{algorithm}
    \floatname{algorithm}{Studienplan}
    \caption{~}
    \textbf{Input: }Liste der Leistungspunkte $l$, Liste der Schwierigkeitsgrade $w$, minimale LP $c$\\
    \textbf{Output: }Studienplan
    \begin{algorithmic}[1]
      \STATE $T \leftarrow$ Tabelle mit $n$ Zeilen und $l_{max}$ Spalten, gefüllt mit $+inf$
      \STATE Für alle Spalten kleiner als $l_0$ Fülle erste Zeile mit $w_0$
      \FOR{$j=1$ \TO $n$} \DO
        \FOR{$i=0$ \TO $l_{max}$} \DO
          \STATE $T[j][i] \leftarrow min(T[j][i-1], T[j-l_i][i-1] + w_i)$
        \ENDFOR
      \ENDFOR
    \end{algorithmic}
  \end{algorithm}\\
  $T[j][i]$ Sei die i-te Spalte und j-te Zeile in der Tabelle T. Falls $i<0$ dann sei der Wert 0.\\
  Den Studienplan ermittelt man wie in e) beschrieben.\\
  \textbf{Laufzeit:}\\
  Da für jede Zelle nur Konstante Laufzeit benötigt wird wird insgeanmt eine Laufzeit von $\mathcal{O}(n*l_{max})$ (anzahl Zellen) benötigt. Das überschreitet die geforderten Kosten nicht.
}
\teil{c}{
  Entferne zunächst die Pflichtveranstaltungen aus den Listen und ziehe die Summe der Leistungspunkte von $c$ ab. Fahre dann wie in b) fort. Zum Schluss müssen die Pflichtveranstaltungen noch dem Plan hinzugefügt werden.
}
\teil{d}{}
\teil{e}{
  Man finde das Feld mit dem kleinsten Wert, welches rechts von Spalte $c$ steht. Dann muss man den "Pfad" zu diesem Feld zurückverfolgen. Dazu fängt man in diesem Feld an und geht jeweils für ein Feld $(i,j)$ wie folgt vor:\\
  Falls Das darüberliegende Feld $(i-1,j)$ den Gleichen Wert hat nimmt man Veranstaltung $i$ nicht mit in den Plan und springt zu Feld $(i-1,j)$. Falls das darüberliegende Feld größer ist, springt man zu dem Feld $(i-1,j-l_i)$ und nimmt die Veranstaltung mit auf.\\
  Dies wiederholt man bis man bei der obersten Zeile oder der linkesten Spalte angelangt ist.
}
}

\loes{3}{
\teil{a, b}{ siehe c)}
\teil{c}{
  Man ordne jeder Farbe folgendermaßen eine Zahl zu:\\
  rot $\rightarrow 0$\\
  grün $\rightarrow 1$\\
  blau $\rightarrow 2$\\
  Im folgenden werde ich die Zahlen stellvertretend für die Farben verwenden.\\
  Der erste Elf errechnet sich die Farbe die er sagt indem er die Summe aller Farben seiner Vorderelfen bestimmt und modulo 3 rechnet. Diese Farbe sei $a \in \{0,1,2\}$.\\
  Der nächste Elf errechnet dann die Summe $S$ all seiner Vorgänger modulo 3. Dann bestimmt er $|a - S|$ mod $3$, welches die Farbe seiner mütze ist und nennt diese.\\
  Alle anderen Elfen erechnen $S$ als die Summe aller genannten Farben (ohne $a$) plus die Summe aller ihrer Vorderelfen, modulo 3. Ihre eigenen Mützenfarben können sie auch wieder mit $|a - S|$ mod $3$ berechnen.\\
  Mit dieser Strategie bekommen alle bis auf den hintersten Elfen garantiert ein Geschenk.
}
}


\end{document}
