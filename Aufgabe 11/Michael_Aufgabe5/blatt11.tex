\documentclass{scrartcl}
\usepackage{amsmath, amssymb, amsthm}
\usepackage{ngerman}
\usepackage[utf8]{inputenc}
\usepackage[T1]{fontenc}
\usepackage{geometry}
\usepackage{color}
\usepackage{graphicx}
\usepackage{algorithm}
\usepackage{hyperref}
\usepackage{listings}
%%\usepackage{algpseudocode} 
\renewcommand*\thealgorithm{}
\usepackage{algorithmic}
\usepackage{pdfpages}
\geometry{top=24mm,textheight=245mm,textwidth=160mm,heightrounded,right=27mm,head=14.5pt}


\renewcommand{\labelenumi}{\alph{enumi})}
\setlength{\parindent}{0mm}

\DeclareMathOperator{\rev}{rev}

\newcommand{\loes}[2]{\medskip\noindent{{\bf Lösung zu Aufgabe #1}\\ {#2}}\vspace{0.5cm}}



\begin{document}
\pagestyle{plain}


\noindent
\begin{minipage}{0.66\textwidth}
Übungen zur Vorlesung\\
Einführung in die Algorithmik WS 15/16\\
~\\
%%%%%%%%%%%%%%%%%% HIER NAME EINTRAGEN! %%%%%%%%%%%%%%%%%%%%%%%%%%%%%
\textbf{Michael Fabian}
%%%%%%%%%%%%%%%%%%%%%%%%%%%%%%%%%%%%%%%%%%%%%%%%%%%%%%%%%%%%%%%%%%%%%
\end{minipage}
~
\begin{minipage}{0.30\textwidth}
Hasso-Plattner-Institut\\
Potsdam\\
\today
\end{minipage}


\begin{center}
 \huge \bf Lösung zu Übungsblatt 11
\end{center}


\loes{5a)}
{
(Jeder Schritt ist auf einer Extraseite)
\includepdf[pages=1-6]{trees.pdf}
}
\loes{5b)}
{
\includepdf[pages=7]{trees.pdf}
}
\end{document}

