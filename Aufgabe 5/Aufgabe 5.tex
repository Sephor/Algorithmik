\documentclass{scrartcl}
\usepackage{amsmath, amssymb, amsthm}
\usepackage{ngerman}
\usepackage[utf8]{inputenc}
\usepackage[T1]{fontenc}
\usepackage{geometry}
\usepackage{color}
\usepackage{graphicx}
\usepackage{algorithm}
\usepackage{hyperref}
%%\usepackage{algpseudocode}
\renewcommand*\thealgorithm{}
\usepackage{algorithmic}
\geometry{top=24mm,textheight=245mm,textwidth=160mm,heightrounded,
right=27mm,head=14.5pt}


\renewcommand{\labelenumi}{\alph{enumi})}
\setlength{\parindent}{0mm}

\DeclareMathOperator{\rev}{rev}

\newcommand{\loes}[2]{\vspace{0.5cm}\\\medskip\noindent{{\bf Lösung zu Aufgabe #1}\\
{#2}}}
\newcommand{\teil}[2]{\vspace{0.3cm}\\\medskip\noindent{{\bf#1)}\\
{#2}}}
\newcommand{\bew}[2]{{\it ZZ. }{#1}\\
{\begin{proof}[Beweis] {#2} \end{proof}}}


\begin{document}
\pagestyle{plain}


\noindent
\begin{minipage}{0.66\textwidth}
Übungen zur Vorlesung\\
Einführung in die Algorithmik WS 15/16\\
~\\
%%%%%%%%%%%%%%%%%% HIER NAME EINTRAGEN! %%%%%%%%%%%%%%%%%%%%%%%%%%%%%
\textbf{Moritz Eissenhauer}
%%%%%%%%%%%%%%%%%%%%%%%%%%%%%%%%%%%%%%%%%%%%%%%%%%%%%%%%%%%%%%%%%%%%%
\end{minipage}
~
\begin{minipage}{0.30\textwidth}
Hasso-Plattner-Institut\\
Potsdam\\
\today
\end{minipage}


\begin{center}
 \huge \bf Lösung zu Übungsblatt XX
\end{center}

\loes{2}{
\teil{a}{
    $I(\sigma)=\{(1,3),(1,7),(3,7),(4,5),(4,7),(4,8),(5,7),(5,8),(6,7),(6,8)\}$
}
\teil{b}{
    $\sigma=n,n-1,n-2,...,1$ hat $\frac{(n-1)(n-2)}{2}$ Fehlstände
}
\teil{c}{
    \bew{
        $I(\sigma)\neq\emptyset\Rightarrow\exists 1 \leq i < n:(i,i+1)\in I(\sigma)$
    }
    {
        Durch widerspruch:\\
        WA: $I(\sigma)\neq\emptyset\wedge\neg\exists 1 \leq i < n:(i,i+1)\in I(\sigma)$
        \begin{equation}
            \begin{split}
                \Rightarrow &\forall 1 \leq i < n:(i,i+1)\notin I(\sigma)\\
                \Rightarrow &\forall 1 \leq i < n:\sigma(i) \leq \sigma(i+1)\\
                \Rightarrow &\forall 1\leq i < j < n:\sigma(i) \leq \sigma(j)\\
                \Rightarrow &\sigma = \{1,2,...,n\}\\
                \Rightarrow &I(\sigma)=\emptyset\\
                & \hookrightarrow $Widerspruch zu WA$
            \end{split}
        \end{equation}
        Da WA zum Widerspruch führ muss {\it ZZ} gelten.
    }
}
\teil{d}{
    \bew{
        Insertsort mit $A=[\sigma(1),\sigma(2),...,\sigma(n)]$ hat laufzeit $\mathcal{O}(n+|I(\sigma)|)$
    }
    {
        \\Beweis Induktiv:\\
        IA: $I(\sigma) = \emptyset \Rightarrow$ Es gibt keine Fehlstände. Das heißt A ist schon sortiert und die Laufzeit von Insertsort ist $T=c_1n+c_2 \in \mathcal{O}(n)$\\
        IV: {\it ZZ} gelte für ein beliebiges $K=I(\sigma)$.\\
        IS: Sei $K'=K \cup \{(k,l)\}$ mit $1\leq k<l\leq n$.\\
        Falls $(k,l) \in K$ ist $K'=K$ und die Laufzeit verändert sich nicht.\\
        Falls $(k,l) \notin K$ verändert sich die Iterationszahl des for-Loops nicht. Die des while-Loops veränders sich auch nicht für $i< l$, da alle Vergleiche gleih ausfallen wie bei $K$. ......
    }
}
}


\end{document}
